% -----------------------------------------------------------------------------
% Oppgave 1
% -----------------------------------------------------------------------------
%<*prob1>
\begin{problem}
  Et musikk digitalt distribuering-system skal konstrueres. Det er en sentral og 30 ulike kjøpesentre som skal benytte systemet. Musikken ligger i området 50-20000 Hz for hver kanal. Vi skal 20 ulike kanaler. Sender og mottakere er koplet i en ensrettet ring. Det finnes en styrekanal med samme kapasitet som en musikkanal på ringen. Konstruer systemet.
\end{problem}
%</prob1>
% -----------------------------------------------------------------------------
% Oppgave 2
% -----------------------------------------------------------------------------
%<*prob2>
\begin{problem}
  Forklar begrepet korteste vei ruting. Finnes det noen varianter?
\end{problem}
%</prob2>
% -----------------------------------------------------------------------------
% Oppgave 3
% -----------------------------------------------------------------------------
%<*prob3>
\begin{problem}
  Et digitalt signal (0V,+5V) skal overføres mellom en sender og en mottaker. Det skal benyttes en linjekode. Velg enkleste metode når:
  \begin{enumerate}[label=\alph*)]
    \item Det ikke er noe krav til signalet
    \item Det kreves at likestrømskomponenten er 0.
    \item Det kreves at synkroniseringsinformasjon ikke går tapt.
    \item Både b og c må være oppfylt.
  \end{enumerate}
\end{problem}
%</prob3>
% -----------------------------------------------------------------------------
% Oppgave 4
% -----------------------------------------------------------------------------
%<*prob4>
\begin{problem}
  Hva er
  \begin{itemize}
    \item Termisk støy
    \item Intermodulasjonsstøy
    \item Krysstale
    \item em impulsstøy
  \end{itemize}
\end{problem}
%</prob4>
% -----------------------------------------------------------------------------
% Oppgave 5
% -----------------------------------------------------------------------------
%<*prob5>
\begin{problem}
  Hva er forskjellen på Nyquist og Shannons formler
\end{problem}

% -----------------------------------------------------------------------------
%</prob5>                            Oppgave 6                        %<*prob6>
% -----------------------------------------------------------------------------

\begin{problem}
  Forklar hovedprinsippet for 4-QAM og 4-PSK. Sammenlikn
\end{problem}
%</prob6>
% -----------------------------------------------------------------------------
% Oppgave 7
% -----------------------------------------------------------------------------
%<*prob7>
\begin{problem}
  Hva er de viktigste parameterne man har for å evaluere linjekoder (digital--digital)

  Følgende bitmønster skal sendes 11001010000001. Tegn hvordan dette blir kodet ved hjelp av:
  \begin{enumerate}
    \item Enkel NRZ-kode
    \item NRZ-L
    \item NRZ-I
    \item Bipolar AMI
    \item Manchester
 \end{enumerate}
\end{problem}
%</prob7>
% -----------------------------------------------------------------------------
% Oppgave 8
% -----------------------------------------------------------------------------
%<*prob8>
\begin{problem}
  Hvilken av kodene over ville du bruke hvis:
  \begin{enumerate}
    \item Du skal ha en enklest mulig kode som ikke har noen spesielle krav.
    \item Et nett hvor det er absolutt krav til at signalet inneholder synkronisering.
  \end{enumerate}
\end{problem}
%</prob8>
% -----------------------------------------------------------------------------
% Oppgave 9
% -----------------------------------------------------------------------------
%<*prob9>
\begin{problem}
  Forklar hvordan du gjenvinner bærebølgen og demodulerer et BPSK signal. Tegn og bruk matematikk.
\end{problem}
%</prob9>
% -----------------------------------------------------------------------------
% Oppgave 10
% -----------------------------------------------------------------------------
%<*prob10>
\begin{problem}
  Konstruer diagram (fase og amlityde) for en 16-QAM modulator hvor du har 8 faser og 4 ulike amplityder til disposisjon.
 \end{problem}
%</prob10>
