    % -*- root: ../main.tex -*-
% -----------------------------------------------------------------------------
%                               Oppgave 1
% ---------------------------------------------------------------------%<*sol1>

%</sol1>-----------------------------------------------------------------------
% Oppgave 2
% ---------------------------------------------------------------------%<*sol2>
Med \emph{korteste} vei kan man mene den fysisk korteste, kjappeste, eller den veien med minst belastning. Altså den veien som gjør at pakken kommer kjappest frem.
%</sol2>-----------------------------------------------------------------------
%                               Oppgave 3
% ---------------------------------------------------------------------%<*sol3>
\begin{enumerate}[label=\alph*)]
    \item Man kan bruke en unipolar
    \item AMI skifter mellom + og - i signalet for ``\emph{1-ere}''.
    \item Manchester koding
    \item Manchster
\end{enumerate}
%</sol3>-----------------------------------------------------------------------
%                               Oppgave 4
% ---------------------------------------------------------------------%<*sol4>
\begin{itemize}
    \item \textbf{Termisk støy} er elektrisk støy som skyldes vibrasjoner av elektroner. Når en elektronisk komponent, f.eks. en motstand, har en temperatur over det absolutte nullpunktet, –273,15$^{\circ}$C, vil de termiske vibrasjonene av elektronene i komponenten gi opphav til elektriske spenninger.

    \item \textbf{Intermodulasjonsstøy} er støy som kommer innenfra. som for eksempel, støy fra strømforsyner.

    \item \textbf{Krysstale} er når signal fra en kilde blir plukket opp av en annen. Et lederpar i en TP--kabel kan plukke opp signalet fra et annet par, sånn at man f.eks hører naboen når man løfter opp røret.

    \item \textbf{EM impulsstøy} er støy forutasket av ytre elektromagnetiske pulse. Et lynnedslag er en typisk kilde for dette.
\end{itemize}
%</sol4>-----------------------------------------------------------------------
%                               Oppgave 5
% ---------------------------------------------------------------------%<*sol5>
    \textbf{Nyquist:} $$C = 2 \cdot B \log_2 \left( M \right)$$

    Nyquists fomrmel gir svar på hvor kanalkabasitet vi har i $bps$ under støyfrie forhold. Noe som i praksis ikke er mulig.\\~\\


    \textbf{Shannon:} $$ C = B \log_2 \left( 1 + \frac{S}{N}\right) $$

    Shannons formel tar utgangspunkt i forholdet mellom \emph{datarate}, støy of \emph{feilrate}. For et gitt støynivå, vil høyere datarate føre til flere feil.\\
    Signal--støy-forholdet blir ofte målt hos mottaker. $(S/N)_{dB} = 10\log_2\frac{\text{signal}}{\text{noise}}$

%</sol5>-----------------------------------------------------------------------
%                               Oppgave 6
% ---------------------------------------------------------------------%<*sol6>
Vi tar utgangspunkt i to måter og sende et to--bits signal analogt.\\

\textbf{4-PSK} sender det analoge signalet med 45$^\circ$ forskyvning pr. bit--verd.\\

\textbf{4-QAM} sender signalet med to amplituder og to faser.\\
\includegraphics[width=.49\textwidth]{4psk.pdf}
\includegraphics[width=.49\textwidth]{4qam.pdf}
%</sol6>-----------------------------------------------------------------------
%                               Oppgave 7
% ---------------------------------------------------------------------%<*sol7>

\begin{figure}[H]
    \includegraphics[width=\textwidth]{src/tikz/bitmonster.pdf}
\end{figure}
%</sol7>-----------------------------------------------------------------------
%                               Oppgave 8
% ---------------------------------------------------------------------%<*sol8>
\begin{enumerate}
    \item Enkel NRZ
    \item Manchester

\end{enumerate}
%</sol8>-----------------------------------------------------------------------
%                               Oppgave 9
% ---------------------------------------------------------------------%<*sol9>
\includegraphics[width=.5\textwidth]{src/tikz/Output/circuit.pdf}
\includegraphics[width=0.5\textwidth]{src/tikz/Output/barebolge.pdf}

%</sol9>-----------------------------------------------------------------------
%                               Oppgave 10
% --------------------------------------------------------------------%<*sol10>
\begin{minipage}{0.49\textwidth}
    \begin{tabular}{ l  c c}
        \toprule
        Bit nr: & Fase & Amplitude\\ \midrule
        0000 & $270^{\circ}$   & 4 \\
        0001 & $225^{\circ}$   & 3 \\
        0010 & $315^{\circ}$   & 3 \\
        0011 & $270^{\circ}$   & 2 \\
        0100 & $225^{\circ}$   & 1 \\
        0101 & $315^{\circ}$   & 1 \\
        0110 & $180^{\circ}$   & 4 \\
        0111 & $180^{\circ}$   & 2 \\
        1000 & $0/360^{\circ}$ & 2 \\
        1001 & $0/360^{\circ}$ & 4\\
        1010 & $135^{\circ}$   & 1 \\
        1011 & $45^{\circ}$    & 1 \\
        1100 & $90^{\circ}$    & 2 \\
        1101 & $135^{\circ}$   & 3 \\
        1110 & $45^{\circ}$    & 3 \\
        1111 & $90^{\circ}$    & 4 \\ \bottomrule
    \end{tabular}
\end{minipage}
\begin{minipage}[L]{0.49\textwidth}
    \includegraphics[width=1.1\textwidth]{src/tikz/qam.pdf}
\end{minipage}
%</sol10>
